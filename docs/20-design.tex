\chapter{Конструкторская часть}

В данном разделе будут рассмотрены требования к программному обеспечению, а также схемы алгоритмов, выбранных для решения поставленной задачи. Также будут описаны пользовательские структуры данных и приведена структура реализуемого программного обеспечения.

\section{Общий принцип работы программы}

На рисунках \ref{img:idef0_0}--\ref{img:idef0_1} приведены IDEF0 диаграммы, которые представляют организацию работы программы.

\img{105mm}{idef0_0}{IDEF0 диаграмма работы программы}

\img{105mm}{idef0_1}{Последовательность действий для визуализации кубика льда с пузырьками воздуха}

\newpage

\section{Требования к ПО}

Программа должна предоставлять доступ к следующему функционалу: перенос, масштабирование и поворот объектов сцены, а также возможность изменять параметры сцены --- количество и характеристики пузырьков воздуха, количество и свойства источников освещения.

К ПО предъявляются следущие требования:
\begin{itemize}
	\item программа должна давать отклик на действия пользователя за комфортное для человека время;
	\item программа должна корректно реагировать на любые действия пользователя.
\end{itemize}

\section{Пересечение луча и сферы}

Уравнение луча представлено ниже:

\begin{equation}
	P = O + t\overrightarrow{D}, t \geq 0,
	\label{eq:ref5}
\end{equation}
где \( P \) -- точка лежащая на луче, \( O \) -- начало луча, $\overrightarrow{D}$ -- направление луча, \( t \) -- произвольное положительное действительное число.

Сфера — это множество точек $P$, лежащих на постоянном расстоянии $r$ от фиксированной точки $C$. Тогда можно записать уравнение, удовлетворяющее этому условию:

\begin{equation}
	distance(P,C) = r
	\label{eq:ref6}
\end{equation}

Запишем расстояние (\ref{eq:ref6}) между P и C как длину вектора из P в C.

\begin{equation}
	|P-C|=r
\end{equation}

Заменим на скалярное произведение вектора на себя:

\begin{equation}
	\sqrt{\langle P - C\rangle, \langle P - C\rangle} = r
\end{equation}

Избавимся от корня:

\begin{equation}
	\langle P - C\rangle, \langle P - C\rangle = r^2
	\label{eq:ref7}
\end{equation}

В итоге есть два уравнения - уравнение луча и сферы. Найдем пересечение луча со сферой. Для этого подставим (\ref{eq:ref5}) в (\ref{eq:ref7})

\begin{equation}
	\langle O + t\overrightarrow{D} - C \rangle, \langle O + t\overrightarrow{D} - C\rangle = r^2
\end{equation}

Разложим скалярное произведение и преобразуем его. В результате получим:

\begin{equation}
	t^2 \langle \overrightarrow{D}, \overrightarrow{D} \rangle + 2t \langle \overrightarrow{OC}, \overrightarrow{D} \rangle + \langle \overrightarrow{OC}, \overrightarrow{OC} \rangle -r^2 = 0
	\label{eq:ref8}
\end{equation}

Представленное квадратное уравнение (\ref{eq:ref8}) имеет несколько возможных случаев решения. Если у уравнения одно решение, то луч касается сферы. Два решения -- луч пересекает сферу. Нет решений -- луч не пересекается со сферой.

\newpage

\section{Схемы алгоритмов}

\subsection{Алгоритм обратной трассировки лучей}

На рисунке \ref{img:rt_sch} представлена схема алгоритма обратной трассировки лучей.

\img{175mm}{rt_sch}{Схема алгоритма обратной трассировки лучей}

\subsection{Алгоритм поиска пересечений луча и сферы}

На рисунке \ref{img:irs_sch} представлена схема алгоритма поиска пересечений луча и сферы.

\img{195mm}{irs_sch}{Схема алгоритма поиска пересечений луча и сферы}

\subsection{Алгоритм поиска пересечений луча и куба}

На рисунке \ref{img:irc_sch} представлена схема алгоритма поиска пересечений луча и куба.

\img{195mm}{irc_sch}{Схема алгоритма поиска пересечений луча и куба}

\section{Описание общей структуры ПО}

Для данной программы следует выделить следующие логические модули: 

\begin{itemize}
	\item \textit{main} --- точка входа в программу и загрузка модуля интерфейса приложения;
	\item \textit{application} --- домен реализации интерфейса программы;
	\item \textit{geometry} --- модуль реализации операций геометрических преобразований;
	\item \textit{model} --- модуль модели сцены;
	\item \textit{objects} --- модуль описания объектов сцены;
	\item \textit{renderer} --- модуль реализации отрисовщика сцены.
\end{itemize}

На рисунке \ref{img:classes} представлена диаграмма классов, отражающая разбиение и взаимодействие объектов сцены.

\newpage

\img{135mm}{classes}{Диаграмма классов разрабатываемого ПО}

\captionsetup{singlelinecheck = false, justification=centering}

\section{Выводы}

В данном разделе были рассмотрены требования к программе, общий алгоритм ее работы, представлены схемы для реализации выбранных алгоритмов и математические формулы.
