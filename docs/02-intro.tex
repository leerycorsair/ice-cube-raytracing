\chapter*{ВВЕДЕНИЕ}
\addcontentsline{toc}{chapter}{ВВЕДЕНИЕ}

Сегодня, в век высоких технологий, роль компьютерной графики велика как никогда. Более того, потребность в реалистичном изображении существует во всех сферах жизни человека \cite{graphics_usage}. 

Во-первых, кинематограф занимает одно из центральных мест в индустрии развлечений, а реалистичные спецэффекты позволяют создавать самые удивительные и невообразимые сцены, что в свою очередь делает кино более зрелищным причем с минимальными затратами. В конечном итоге массовый потребитель имеет более широкий выбор фильмов для просмотра, а цена остается минимальной. 

Во-вторых, в игровой индустрии графика, становясь все более реалистичной, обеспечивает полное погружение пользователя в игровой процесс и создает изображение, едва отличимое от реальности. 

В-третьих, реалистичная компьютерная графика упрощает работу современным архитекторам и дизайнерам: можно в мельчайших деталях рассмотреть объекты, а также достичь более быстрой разработки проектов и обеспечить тесное взаимодействие между заказчиком и исполнителем. 

В-четвертых, современные технологии компьютерной графики находят все большее применение в дополненной реальности. Например, в сложных хирургических операциях врачи используют специальные очки, показывающие капилляры с высокой долей точности, что позволяет избежать лишних кровопотерь.

Таким образом, задача создания реалистичного изображения актуальна и крайне востребована. Наиболее значимой составляющей этой задачи является корректный расчет освещения. В окружающем нас мире способны видеть различные вещи благодаря тому, что лучи света преломляются, излучаются, отражаются от объектов и попадают на нашу сетчатку глаза. Следовательно, необходимо подобрать методы, моделирующие тот же самый процесс, но уже в компьютерной визуализации.

Целью данной работы является проектирование программного обеспечения, реализующего построение реалистичного изображения с учетом оптических свойств объекта, на примере визуализации кубика льда, имеющего пузырьки воздуха внутри. 

\newpage

Для достижения указанной выше цели следует выполнить следующие задачи:
\begin{itemize}
	\item рассмотреть существующие решения для моделирования;
	\item формально описать структуру объектов синтезируемой схемы;
	\item выбрать или модифицировать существуюшие алгоритмы трехмерной графики, необходимые для визуализации модели;
	\item привести схемы выбранных алгоритмов для решения поставленной задачи;
	\item описать используемые типы данных;
	\item описать структуру разрабатываемого ПО;
	\item реализовать выбранные алгоритмы визуализации;
	\item протестировать разработанное ПО;
	\item сделать выводы на основании проделанной работы.
\end{itemize}

