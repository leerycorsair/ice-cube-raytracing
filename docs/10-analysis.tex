\chapter{Аналитическая часть}

В этом разделе будут рассмотрены предметная область и основные алгоритмы,
необходимые для создания реалистичного изображения, а также произведён
и обоснован выбор алгоритмов для реализации в проекте.

\section{Постановка задачи}

Задачей программы является представление трехмерной визуализации кубика льда с пузырьками воздуха, которая позволила бы наблюдать процесс преломления, благодаря чему было бы проще изучить оптические свойства льда и основные законы движения световых лучей.

\section{Предметная область}

Лед - вода в твердом агрегатном состоянии. Льдом иногда называют некоторые вещества в твердом агрегатном состоянии, которым свойственно иметь жидкую или газообразную форму при комнатной температуре; в частности сухой лед, амиачный лед или метановый лед \cite{icenature}.

В настоящее время известны три аморфных разновидности и 17 кристаллических модификаций льда.  В природных условиях Земли вода образует кристаллы одной кристаллической модификации — гексагональной сингонии. Лед встречается в природе в виде собственно льда (материкового, плавающего, подземного), а также в виде снега, инея, изморози. Под действием собственного веса лед приобретает пластические свойства и текучесть.

Природный лед обычно значительно чище, чем вода, так как при кристаллизации воды в первую очередь в решётку встают молекулы воды. Лед может содержать механические примеси — твёрдые частицы, капельки концентрированных растворов, пузырьки газа (см. рисунок \ref{img:real_ice}). Наличием кристалликов соли и капелек рассола объясняется солоноватость морского льда.

\boximg{60mm}{real_ice}{Фото естественных кубиков льда с пузырьками воздуха}

\section{Обзор сущестующих программных решений}

Следует отметить, что существуют приложения, которые позволяют выполнить визуализацию льда. Более того, некоторые из также моделируют такие физические процессы, как таяние, трескание и кристаллизацию с течением времени.

Одним из самых популярных решений является Blender \cite{blenderice}. Данная программа позволяет выполнить высоко детализированную визуализацию с любыми условиями окружающей среды (см. рисунки \ref{img:blender_ice_cube_with_bubbles}--\ref{img:blender_ice_with_cracks}).

\boximg{60mm}{blender_ice_cube_with_bubbles}{Моделирование кубика льда с пузырьками воздуха в Blender}

\newpage

\boximg{60mm}{blender_ice_melting}{Моделирование таяния льда в Blender}

\boximg{60mm}{blender_ice_with_cracks}{Моделирование трескания льда в Blender}

На рисунке \ref{img:keyshot_cocktails} показан пример кубика льда из рекламных материалов, созданных в программе KeyShot. Данная утилита предназначенна для визуализации дизайнерских решений и не обладает широким функционалом настройки свойств окружения.

\boximg{60mm}{keyshot_cocktails}{Визуализация ледяных коктелей в KeyShot}

Также широко используется моделирование в программе AutoDesk Maya ввиду совместимости с другими продуктами компании и легкого импорта или экспорта модели (см. рисунок \ref{img:maya_ice}).

Список программ для визуализации не исчерпывается приведенными выше решениями. Их существует гораздо больше, каждое из которых имеет свои преимущества и недостатки в конкретной задаче.

\boximg{60mm}{maya_ice}{Визуализация льда в Maya}

\section{Формализация объектов синтезируемой сцены}

Сцена состоит из следующих объектов:
\begin{itemize}
	\item трехмерная модель кубика льда, которая описывается положением в пространстве и оптическими свойствами;
	\item трехмерные пузырьки воздуха внутри кубика льда --- в рамках поставленной задачи представляют собой сферические объекты. Каждый пузырек имеет свой размер;
	\item источниками света являются материальные точки, испускающие свет. Положение источника задается трехмерными координатами, а направление распространения вектором.
\end{itemize}

\section{Анализ способов задания трехмерных моделей}

В компьютерной графике в основном используются 3 вида моделей трехмерных объектов \cite{models}:

\begin{enumerate}
	\item \textbf{Каркасная(проволочная) модель}. Это простейший вид моделей, содержащий минимум информации --- о вершинах и рёбрах объектов. Главный недостаток --- такая модель не всегда правильно передает представление об объекте (например, если в объекте есть отверстия).
	\item \textbf{Поверхностная модель}. Отдельные участки задаются как участки поверхности того или иного вида. Эта модель решает проблему каркасной, но все еще имеет недостаток --- нет информации о том, с какой стороны поверхности находится собственный материал.
	\item \textbf{Объемная модель}. В отличии от поверхностной, содержит указание расположения материала (чаще всего указанием направления внутренней нормали).
\end{enumerate}

Таким образом, каркасная модель не подойдет для решения поставленной задачи, так как она не позволяет построить реалистичное изображение из-за недостатка информации, а объемная модель будет избыточной, так как нам не нужна информация о том, с какой стороны находится материал, из которого изготовлен тот или иной объект. Следовательно, можно сделать вывод о том, что для решения поставленной задачи подойдет именно поверхностная модель объекта. 

\newpage

\section{Анализ способов задания поверхностных моделей}

Можно выделить следующие способы задания поверхностных моделей:

\begin{enumerate}
	\item \textbf{Аналитический}. 
	
	Этот способ подразумевает описание поверхности с помощью функции, неявного уравнения или параметрической формы. Для описания сложных поверхностей в этом случае можно использовать сплайны для аппроксимации их отдельных фрагментов. Чаще всего в трехмерной графике используют кубические сплайны, т.к. это наименьшая из степеней, позволяющая описать любую форму и обеспечить непрерывную первую производную при стыковке сплайнов, благодаря чему поверхность будет без изломов в местах стыка. 
	
	К достоинствам аналитической модели можно отнести легкость расчета координат каждой точки поверхности, нормали, а также небольшой объем данных для описания достаточно сложных форм. Среди недостатков - сложность формул и используемых для их реализации функций, значительно снижающих скорость выполнения операций отображения, а также невозможность в большинстве случаев применить данный способ непосредственно для изображения поверхности - поверхность отображается как многогранник, координаты вершин и граней которого рассчитываются в процессе отображения, что уменьшает скорость в сравнении с полигональной моделью описания.
	
	\item \textbf{Полигональная сетка}.
	Полигональная сетка представляет собой описание объекта совокупностью его вершин, ребер и граней (полигонов). Рассмотрим различные способы хранения информации о полигональной сетке:
	
	\begin{itemize}
		\item Вершинное представление содержит информацию о соседних вершинах для каждой вершины объекта. При таком способе хранения нелегко выполняются операции с ребрами и гранями в связи с тем, что информация о них выражена неявно. Однако данный метод не требует большого объема памяти и позволяет эффективно производить трансформации.
		\item Список граней описывает объект как множество граней и множество вершин. В таком случае, в отличие от вершинного представления, и грани, вершины объекта явно заданы, так что операции поиска соседних граней и вершин постоянны по времени. Однако ребра все еще не заданы явно, так что некоторые операции будут проводиться неэффективно по времени, например, поиск граней, окружающих заданную, а также разрыв или объединение грани.
		\item Крылатое представление уже хранит явно информацию и о вершинах, и о гранях, и о ребрах объекта. Такой метод решает проблему неэффективности операций разрыва и объединений, однако при этом требует большого объема памяти и достаточно сложен из-за содержания множества индексов.
	\end{itemize}	
\end{enumerate}

Для визуализации кубика льда и пузырьков воздуха удобно использовать следующие геометрические примитивы: куб и сфера, которые в свою очередь легко описать аналитическими уравнениями. Таким образом, выбран аналитический способ задания поверхностей.

\section{Анализ алгоритмов удаления невидимых ребер и поверхностей}

Алгоритмы удаления невидимых линий и поверхностей служат для определения линий ребер, поверхностей, которые видимы или невидимы для наблюдателя, находящегося в заданной точке пространства.

Решать задачу можно в: 
\begin{itemize}
	\item объектном пространстве --- используется мировая система координат,
	достигается высокая точность изображения. Обобщенный подход, основанный на анализе пространства объектов, предполагает попарное сравнение положения всех объектов по отношению к наблюдателю;
	\item пространстве изображений --- используется экранная система координат, связанная с устройством, в котором отображается результат (графический дисплей).
\end{itemize}

Под экранированием подразумевается загораживание одного объекта другим. Под глубиной подразумевается значение координаты Z, направленной от зрителя, за плоскость экрана.

\subsection{Алгоритм Робертса}

Алгоритм Робертса решает задачу удаления невидимых линий. Работает в объектном пространстве. Данный алгоритм работает исключительно с выпуклыми телами. Если тело изначально является не выпуклым, то нужно его разбить на выпуклые составляющие. Алгоритм целиком основан на математических предпосылках \cite{roberts}.

Из-за сложности математических вычислений, используемых в данном
алгоритме, и из-за дополнительных затрат ресурсов на вычисление матриц
данный алгоритм является медленным.

\subsection{Алгоритм Z-буфера}
Алгоритм Z–буфера позволяет определить, какие пикселы граней сцены видимы, а какие заслонены гранями других объектов \cite{zbuffer}. Z–буфер --- это двухмерный массив, его размеры равны размерам окна, таким образом, каждому пикселу окна соответствует ячейка Z-буфера. В этой ячейке хранится
значение глубины пиксела. Перед растеризацией сцены Z–буфер заполняется значением, соответствующим максимальной глубине. В случае, когда глубина характеризуется значением w, максимальной глубине соответствует нулевое значение. Анализ видимости происходит при растеризации граней, для каждого пиксела рассчитывается глубина и сравнивается со значением в Z–буфере, если рисуемый пиксел ближе (его w больше значения в Z–буфере), то пиксел рисуется, а значение в Z–буфере заменяется его глубиной. Если пиксел дальше, то пиксел не рисуется и Z–буфер не изменяется, текущий пиксел дальше того, что нарисован ранее, а значит невидим.

Недостатками данного алгоритма являются довольно большой объем требуемой памяти, трудоемкость устранения лестничного эффекта и реализации эффектов прозрачности.

\subsection{Алгоритм прямой трассировки лучей}

Основная идея алгоритма прямой трассировки лучей состоит в том, что наблюдатель видит объекты благодаря световым лучам, испускаемым некоторым источником, которые падают на объект, отражаются, преломляются или проходят сквозь него и в результате достигают зрителя \cite{ray-tracing}.

Основным недостатком алгоритма является излишне большое число рассматриваемых лучей, приводящее к существенным затратам вычислительных мощностей, так как лишь малая часть лучей достигает точки наблюдения. Данный алгоритм подходит для генерации статических сцен и моделирования зеркального отражения, а так же других оптических эффектов.

\subsection{Алгоритм обратной трассировки лучей}
Алгоритм обратной трассировки лучей отслеживает лучи в обратном
направлении (от наблюдателя к объекту)\cite{ray-tracing}. Такой подход призван повысить
эффективность алгоритма в сравнении с алгоритмом прямой трассировки
лучей. Обратная трассировка позволяет работать с несколькими источниками
света, передавать множество разных оптических явлений.

Пример работы данного алгоритма приведен на рисунке \ref{img:rt_scheme}.

\boximg{60mm}{rt_scheme}{Пример работы алгоритма обратной трассировки лучей}

Соответственно, порядок алгоритма трассировки такой: найти точку пересечения луча от наблюдателя с первым объектом, для определения степени освещенности объекта проверить его пересечения со всеми источниками света, рекурсивно вызвать процедуру в случае материала с полупрозрачной и/или отражающей поверхностью.

Считается, что наблюдатель расположен на положительной полуоси z в бесконечности, поэтому все световые лучи параллельны оси z. В ходе работы испускаются лучи от наблюдателя и ищутся пересечения луча и всех объектов сцены. В результате пересечение с максимальным значением z является видимой частью поверхности и атрибуты данного объекта используются для
определения характеристик пикселя, через центр которого проходит данный световой луч.

Для расчета эффектов освещения сцены проводятся вторичные лучи от точек пересечения ко всем источникам света. Если на пути этих лучей встречается непрозрачное тело, значит, данная точка находится в тени.

Несмотря на более высокую эффективность алгоритма в сравнении с прямой трассировкой лучей, данный алгоритм считается достаточно медленным, так как в нем происходит точный расчет сложных аналитических выражений для нахождения пересечения с рассматриваемыми объектами.

\subsection{Выводы}

Оценив все изложенные выше алгоритмы, можно сделать вывод, что для данной работы лучше всего подходит алгоритм обратной трассировки лучей, так как он точно отражает суть физических явлений, таких как отражение и преломление лучей.

\section{Анализ модели освещения}

Для создания реалистичного изображения в компьютерной графике применяются различные алгоритмы освещения. Модель освещения предназначена для расчета интенсивности отраженного к наблюдателю света в каждой точке изображения.

Модель освещения может быть:
\begin{itemize}
	\item локальной --- в данной модели учитывается только свет от источников и
	ориентация поверхности;
	\item глобальной --- в данной модели, помимо составляющих локальной, учитывается еще и свет, отраженный от других поверхностей или пропущенный через них.
\end{itemize}

Локальная модель включает 3 составляющих:

\begin{itemize}
	\item диффузную составляющую отражения;
	\item отражающую составляющую отражения;
	\item рассеянное освещение.
\end{itemize}

Выбор алгоритма построения теней напрямую зависит от выбора алгоритма отсечения невидимых ребер и поверхностей, а также --- от выборамодели освещения.

В алгоритме трассировки лучей тени получаются без дополнительных вычислений за счет выбранной модели освещения, а в алгоритме с Z буфером, можно получить тени, используя второй буфер, полученный подменой точки наблюдения на точку источника света.

Выбрана глобальная модель освещения, так как локальная дает неправильный результат в случае явления преломления света, которое широко используется в данной работе.

\newpage

\section{Выводы}
В данном разделе была рассмотрена предметная область, аналоги разрабатываемому приложению, основные алгоритмы, необходимые для создания реалистичного изображения. 

Подводя итог, следует отметить, что:
\begin{itemize}
	\item для задания трехмерной моделей будет использоваться поверхностная модель;
	\item был выбран аналитической способ задания поверхностной модели;
	\item алгоритм обратной трассировки лучей является наиболее подходящим для удаления невидимых ребер и поверхностей;
	\item будет использоваться глобальная модель освещения.
\end{itemize}


