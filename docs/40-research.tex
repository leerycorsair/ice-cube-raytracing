\chapter{Исследовательская часть}

В данном разделе приведено описание планирования эксперимента и их результаты.

\section{Технические характеристики}

Технические характеристики устройства, на котором выполнялось тестирование:

\begin{itemize}
	\item Операционная система: Windows 10 64-bit \cite{windows}.
	\item Память: 16 GB.
	\item Процессор: AMD Ryzen 5 4600H \cite{amd} @ 3.00 GHz.
\end{itemize}

\section{Постановка эксперимента}

Целью эксперимента является определение зависимости времени генерации изображения алгоритмом обратной трассировки лучей от количества используемых потоков.

Во время тестирования устройство было подключено к блоку питания и не нагружено никакими приложениями, кроме встроенных приложений окружения, окружением и системой тестирования. Оптимизация компилятора была отключена.

В качестве бенчмарка была выбрана сцена состоящая из кубика льда, пяти пузырьков воздуха и одного источника света (максимальная глубина рекурсии 5).

По результатам эксперимента составляются сравнительные таблицы, а также строятся графики зависимостей.

\captionsetup{singlelinecheck = false, justification=raggedright}

\newpage

\section{Результаты эксперимента}

В таблице \ref{tabular:time} приведены результаты эксперимента.

\begin{table}[h!]
	\begin{center}
		\caption{\label{tabular:time}Время работы программы}
		\begin{tabular}{|c|c|}
			\hline
			Количество потоков &  Время обработки сцены \\\hline
				1 & 953\\\hline
				2 & 453\\\hline
				4 & 247\\\hline
				8 & 121\\\hline
				12 & 107\\\hline
				16 & 117\\\hline
				24 & 109\\\hline
				32 & 113\\\hline
		\end{tabular}
	\end{center}
\end{table}

На рисунке \ref{plt:time} представлены результаты эксперимента в графическом виде.

\begin{figure}[h]
	\centering
	\begin{tikzpicture}
		\begin{axis}[
			axis lines=left,
			xlabel=Количество потоков,
			ylabel={Время, мс},
			legend pos=north west,
			ymajorgrids=true
			]
			\addplot table[x=len,y=time,col sep=comma] {inc/csv/time.csv};
		\end{axis}
	\end{tikzpicture}
	\captionsetup{justification=centering}
	\caption{Время работы программы}
	\label{plt:time}
\end{figure}


\captionsetup{singlelinecheck = false, justification=centering}

\section{Выводы}

По результатам эксперимента можно сделать вывод о том, что время вычисления алгоритма обратной трассировки лучей уменьшается линейно с ростом числа потоков. Однако следует отметить, что количество физических ядер ограничивает рост быстродействия. Это подверждается тем, что в современных графических ускорителях десятки тысяч ядер для вычисления кадра изображения в режиме реального времени. 

На основе этих данных с целью дальнейшего улучшения продукта можно также провести анализ, какое количество пузырьков является оптимальным, чтобы изображение было наиболее комфортным для человеческого глаза и при этом было достаточно маленькое время отклика приложения на действия пользователя.



