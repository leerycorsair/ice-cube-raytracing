\chapter*{ЗАКЛЮЧЕНИЕ}
\addcontentsline{toc}{chapter}{ЗАКЛЮЧЕНИЕ}

Цель курсовой работы достигнута --- было реализовано программное обеспечение для визуализации модели кубика льда с пузырьками воздуха. Пользователь может динамически просматривать сцену: масштабировать, переносить и поворачивать модель, а также изменять параметры сцены --- количество и характеристики пузырьков воздуха, количество и свойтва источников освещения.

В ходе выполнения работы были проанализированны существующие алгоритмы компьютерной графики --- были рассмотрены методы задания моделей, способы описания поверхностой модели, способы удаления невидимых ребер и поверхностей, модели освещения, а также были выявлены их преимущества и недостатки.

Также над программным продуктом был проведен эксперимент, определяющий зависимость времени генерации изображения алгоритмом обратной трассировки лучей от количества используемых потоков. результатам эксперимента можно сделать вывод о том, что время вычисления алгоритма обратной трассировки лучей уменьшается линейно с ростом числа потоков. Однако следует отметить, что количество физических ядер выступает ограничивает рост быстродействия.

В дальнейшем этот продукт можно улучшить --- использовать реалистичные текстуры для изображения поверхностей льда и воздушных пузырьков, повысить быстродействие использованием механизмов распараллеливания вычислений алгоритма обратной трассировки лучей, добавить возможность возможность просмотра изменения состояния льда под действием различных физических факторов и т.д.


