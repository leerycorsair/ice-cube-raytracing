\chapter{Технологическая часть}

В данном разделе описаны средства реализации, приведены листинги кода и рассмотрен пользовательский интерфейс.

\section{Средства реализации}

Для решения поставленной задачи был выбран язык программирования C++ ввиду ряда причин:
\begin{itemize}
	\item поддержка объектно-ориентированного подхода к программированию дает возможность создавать четко структурированные и легко модифицируемые программы;
	\item строгая типизация и стандартизация позволяет избежать множество ошибок, перекладывая ответственность за контролем типов компилятору;
	\item будучи представителем компилируемых языков, обеспечивает высокую скорость исполнения, что особенно важно в трудоемких алгоритмах трехмерной компьютерной графики.
\end{itemize}

\section{Листинг кода}

\captionsetup{singlelinecheck = false, justification=raggedright}

В листингах \ref{lst:render1}--\ref{lst:render2} приведена общая реализация алгоритма обратной трассировки лучей.

\begin{lstinputlisting}[
	caption={Алгоритм трассировки лучей (часть 1)},
	label={lst:render1},
	style={go},
	linerange={1-8},
	]{./inc/imp/render.txt}
\end{lstinputlisting}

\begin{lstinputlisting}[
	caption={Алгоритм трассировки лучей (часть 2)},
	label={lst:render2},
	style={go},
	linerange={9-13},
	]{./inc/imp/render.txt}
\end{lstinputlisting}

В листинге \ref{lst:ray_cast} приведена реализация алгоритма пуска луча.

\begin{lstinputlisting}[
	caption={Алгоритм пуска луча},
	label={lst:ray_cast},
	style={go},
	]{./inc/imp/ray_cast.txt}
\end{lstinputlisting}

В листингах \ref{lst:sphere1}--\ref{lst:sphere2} приведена реализация алгоритма поиска пересечения луча и сферы.

\begin{lstinputlisting}[
	caption={Алгоритм поиска пересечения луча и сферы (часть 1)},
	label={lst:sphere1},
	style={go},
	linerange={1-5},
	]{./inc/imp/sphere.txt}
\end{lstinputlisting}

\begin{lstinputlisting}[
	caption={Алгоритм поиска пересечения луча и сферы (часть 2)},
	label={lst:sphere2},
	style={go},
	linerange={6-17},
	]{./inc/imp/sphere.txt}
\end{lstinputlisting}

В листингах \ref{lst:cube1}--\ref{lst:cube2} приведена реализация алгоритма поиска пересечения луча и куба.

\begin{lstinputlisting}[
	caption={Алгоритм поиска пересечения луча и куба (часть 1)},
	label={lst:cube1},
	style={go},
	linerange={1-28},
	]{./inc/imp/cube.txt}
\end{lstinputlisting}

\newpage

\begin{lstinputlisting}[
	caption={Алгоритм поиска пересечения луча и куба (часть 2)},
	label={lst:cube2},
	style={go},
	linerange={29-36},
	]{./inc/imp/cube.txt}
\end{lstinputlisting}


\captionsetup{singlelinecheck = false, justification=centering}

\section{Описание интерфейса программы}

На рисунке \ref{img:ui} представлен интерфейс программы.

Можно выделить следующие возможности интерфейса:
\begin{itemize}
	\item выбор объектов сцены и перемещение объектов сцены;
	\item масштабирование объектов сцены;
	\item задание коэффициентов диффузного, зеркального, рефракционного отражений, блеска и рефракции;
	\item удалять и создавать новые пузырьки;
	\item удалять, создавать и перемещать источники освещения;
	\item регулировать максимальную глубину рекурсии;
	\item изменять позицию, направление камеры, ее угол обзора.
\end{itemize}

\img{85mm}{ui}{Интерфейс программы}

\section{Примеры работы программы}

На рисунках \ref{img:ice_1}--\ref{img:ice_3} приведены результаты работы разработанного ПО, визуализирующего кубик льда с пузырьками воздуха. В каждом из примеров использовались различные оптические параметры, описывающие модель, максимальная глубина рекурсии алгоритма обратной трассировки лучей составляла 5.

\img{85mm}{ice_1}{Пример работы программы (1)}

\clearpage

\img{85mm}{ice_2}{Пример работы программы (2)}

\img{85mm}{ice_3}{Пример работы программы (3)}


\section{Выводы}

В данном разделе были описаны средства реализации, приведены листинги кода и рассмотрен пользовательский интерфейс.
